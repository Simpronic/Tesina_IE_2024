\chapter{RBD}
RBD sta per \textbf{Reliability Block Diagram} ed è una tecnica per modellare sistemi che sono \textbf{Non-state space}.\\
Tali modelli ci permettono di ottenere una soluzione veloce attraverso l'assunzione che i componenti del sistema siano indipendenti.\\
Dunque vediamo il sistema come una serie di componenti interconnessi tra loro e abbiamo informazioni riguardo:
\begin{itemize}
    \item Probabilità di fallimento
    \item Failure rate
    \item Una distribuzione del tempo al fallimento 
    \item Stady state e istantanea unavailability
\end{itemize}
Le assunzioni di tale tecnica sono:
\begin{itemize}
    \item I failure nei moduli sono indipendenti
    \item Se un modulo fallisce rimarrà nel suo stato di fallimento
    \item Un sistema è fallito se non provvede ad un set minimo di moduli funzionali
\end{itemize}
Di sequito sono risolti degli esercizi riguardanti gli RBD.
\section{Esercizio 1}
\begin{figure}[H]
    \centering
    \includegraphics[scale=0.5]{img/chap_5/Ex1.png}
    \caption{Esercizio 1}
    \label{fig:Ex_1}
\end{figure}
\noindent
In prima analisi notiamo i primi collegamenti parallelo serie(denotiamo con || un parallelo e con - una serie), in particolare abbiamo \textbf{C1||C2} e \textbf{C5-C6}
Da cui avremo:
\begin{center}
    $
        R_{C1||C2} = 1−(1−R)(1R) = 1−(1−R)^2 
        R_{C5-C6} = R^2
    $
\end{center} 
Il nuovo schema sarà:
\begin{figure}[H]
    \centering
    \includegraphics[scale=0.5]{img/chap_5/Ex1_1.png}
    \caption{Schema trasformato}
    \label{fig:Ex_1_1}
\end{figure}
\noindent
Non potendo identificare altre strutture serie o parallelo dobbiamo procedere con il teorema del condizionamento su C4.\\
Per prima cosa possiamo ricavare un upper bound con la tecnica dei success path che sono:
\begin{itemize}
    \item C1||C2 \textminus{} C4
    \item C3 \textminus{} C4
    \item C3 \textminus{} (C5 \textminus{} C6)
\end{itemize}
Otteniamo dunque il seguente Upper-bound:
\begin{center}
    $
        R_{sys} \leq 1-(1-R^2)^3(1-R^3)
    $
\end{center}
Essendoci ricavati un upper bound possiamo procedere con l'analisi attraverso il condizionamento.\\
Per il teorema del condizionamento ho:\\
\begin{center}
    $R_{sys}$ = $R*P$(sistema funziona | C è UP) $+ (1-R) \cdot $P(sistema funziona|C è DOWN)
\end{center}
Di seguito riportiamo i due schemi uno con C4 DOWN e l'altro con C4 UP.\\
\begin{figure}[H]
    \centering
    \includegraphics[scale=0.5]{img/chap_5/Ex_1_2.png}
    \caption{C4 UP}
    \label{fig:Ex_1_2}
\end{figure}
\begin{figure}[H]
    \centering
    \includegraphics[scale=0.5]{img/chap_5/Ex_1_3.png}
    \caption{C4 DOWN}
    \label{fig:Ex_1_3}
\end{figure}
\noindent
Riguardo la figura \ref{fig:Ex_1_3} abbiamo C3-(C5-C6) che danno la seguente reliability:
\begin{center}
 P(sistema funziona | C è UP) =  $R^3$
\end{center}
Invece riguardo la figura \ref{fig:Ex_1_2} si può dimostrare, riapplicando il teorema del condizionamento che abbiamo il seguente risultato:\\
\begin{center}
    P(sistema funziona | C è UP) =  $1-(1-R)^3$
\end{center}
Andando a sostituire nella formula del teorema del condizionamento avrò:\\
\begin{center}
     $R_{sys}$ = $R[1-(1-R)^3] + R^3(1-R)$
\end{center}
Avendo un reliability di tipo esponenziale con failure rate $\lambda$, sostituendo ho:\\
\begin{center}
    $R_{sys}(t) = -2e^{-3\lambda t } + 3e^{-2\lambda t}$
\end{center}
Volendo calcolare l'MTTF avrò:
\begin{center}
    $
        \int_0^{\infty} R_{sys}(t) dt = \frac{5}{6\lambda}
    $
\end{center}
\section{Esercizio 2}

\begin{figure}[H]
    \centering
    \includegraphics[scale=0.5]{img/chap_5/Ex2.png}
    \caption{Esercizio 2}
    \label{fig:Ex_2}
\end{figure}
\noindent
Da una prima analisi qualitativa si può notare che in termini di reliability il secondo sistema è più reliable del primo avendo un numero di \textbf{success path} maggiore.\\
Il primo sistema è composto dal parallelo di \textit{m} catene composte da una serie di \textit{s} sistemi.\\
In termini di reliability avremo per il primo sistema la seguente espressione di reliability:
\begin{center}
    $
        R_{sys1} = [1-(1-R^s)^m]
    $
\end{center}
Il secondo sistema invece è composto da una serie di \textit{s} paralleli di \textit{m} sottosistemi.\\
Avremo dunque la seguente reliability:
\begin{center}
    $
        R_{sys2} = [1-(1-R)^m]^s
    $
\end{center}
Per i valori dell'esercizio avrò:
\begin{center}
    $
    R_{sys1} = [1-(1-R^4)^2] 
    R_{sys2} = [1-(1-R)^2]^4 
    $
\end{center}
Per effettuare la comparazione tra i due sistemi dobbiamo fissare il failure rate $\lambda$ e consideriamo R = $e^{-\lambda t}$.
Il parametro failure rate lo prendiamo dall'MTTF che è stato posto a 800h.
Ricordando che MTTF = $\frac{1}{\lambda}$ nel particolare caso di reliability esponenziale, dunque $\lambda = \frac{1}{800h}$. 
Di seguito riportiamo il plot della reliability in funzione del tempo:

\begin{figure}[H]
    \centering
    \includegraphics[scale=0.5]{img/chap_5/plt_rel.png}
    \caption{plot reliability}
    \label{fig:plt_rel}
\end{figure}
\noindent
Dal grafico \ref{fig:plt_rel} si può notare che il sistema2, a prescindere dal valore del mission time, risulta essere drasticamente più reliable del primo.\\
Se supponiamo un mission time di 350h possiamo notare che il primo sistema ha una reliability di circa il 32\% mentre il secondo di circa il 58\%.
Fissato questo mission time per rendere equivalenti i due sistemi modificando il primo si può lavorare sul numero di catene \textit{m} imponendo la seguente equazione:
\begin{center}
    $
    1-(1-R^4)^m = \frac{58}{100}
    $
\end{center}
Da cui possiamo ottenere: 
+\begin{center}
    $
    m = \lceil \log_{1-\exp^{-\frac{4 \cdot 350}{800}}}(1-\frac{58}{100})\rceil = 5
    $
\end{center}
Si può dimostrare che non si può agire sul parametro \textit{s} per migliorare lo schema.\\
Di seguito si riporta l'andamento grafico che conferma circa l'equivalenza tra i due sistemi per un mission time di 300h. 

\begin{figure}[H]
    \centering
    \includegraphics[scale=0.6]{img/chap_5/plt_rel_2.png}
    \caption{plot confronto}
    \label{fig:plt_rel2}
\end{figure}
\noindent

\section{Esercizio 3}
\begin{figure}[H]
    \centering
    \includegraphics[scale=0.6]{img/chap_5/ex_3.png}
    \caption{Esercizio 3}
    \label{fig:ex_3}
\end{figure}
\noindent
Come riportato dalla traccia, abbiamo una topologia skip-ring la quale è una topologia che ci permette il funzionamento anche in presenza di di fallimenti, in particolare la rete funziona se non falliscono due nodi adiacenti.\\
Il sistema presentato può essere assunto come un sistema M-out-of-N, ovvero sistemi composti da N componenti, ciascuno con una certa reliability R, dei quali almeno M devono funzionare.\\
La formula della reliability per questi sistemi è la seguente:
\begin{center}
    $
    R_{MN} = \sum^{N-M}_{i=0} \binom{N}{i} \cdot R^{N-i} \cdot (1-R)^i
    $
\end{center}
Per studiare questo caso però dobbiamo sottrarre i casi in cui la rete non funziona, e quindi assumere un certo numero di fallimenti ed enumerare le possibili combinazioni che portano al fallimenti di nodi adiacenti.\\
Enumerando i vari casi si può vedere che per un numero di fallimenti i $\geq$ 5 la rete non funziona poichè c'è almeno una coppia di nodi adiacenti che non funziona.\\
Enumerando le varie situazioni abbiamo le seguenti casistiche di reliability:
\begin{itemize}
    \item i = 0 se tutti i nodi funzionano correttamente, il sistema funziona ed avrò dunque:
    \begin{center}
        $
        \binom{8}{0} \cdot R^{8} \cdot (1-R)^0 = R^8
        $
    \end{center}
    \item i=1, se fallisce un solo nodo, il sistema continua a funzionare per la caratteristica della topologia, dunque avrò:
        \begin{center}
        $
        \binom{8}{1} \cdot R^{7} \cdot (1-R)^1 = 8R^7 \cdot (1-R)
        $
        \end{center}
    \item i = 2, in questo caso potremmo avere dei fallimenti se falliscono due nodi adiacenti. Si può vedere che le combinazioni possibili sono 8 enumerandole, allora avrò
        \begin{center}
        $
        (\binom{8}{2}-8) \cdot R^{6} \cdot (1-R)^2 = 20R^6 \cdot (1-R)^2
        $
        \end{center}
       \item i = 3 abbiamo due casi in cui il sistema non funziona
       \begin{itemize}
           \item falliscono 3 nodi adiacenti: 8 combinazioni di non funzionamento
           \item falliscono 2 nodi adiacenti e 1 non adiacente: in tale caso abbiamo che ogni coppia di nodi adiacenti ha 4 nodi non adiacenti che possono fallire, dunque 8*4 combinazioni, ovvero 32 
       \end{itemize}
       Per cui:
        \begin{center}
        $
        (\binom{8}{3}-(8+32)) \cdot R^{5}*(1-R)^3 = 16R^5 \cdot (1-R)^3
        $
        \end{center}
    \item i = 4, in questo caso è utile valutare le condizione di funzionamento che quelle di malfunzionamento, in particolare le condizioni di funzionamento essendo 8 nodi, i nodi falliti devono alternarsi, in tale caso abbiamo solo due combinazioni di funzionamento, dunque:
        \begin{center}
        $
        2 \cdot R^4(1-R)^4
        $
        \end{center}
\end{itemize}
La reliability del sistema è la somma di quelle sopra riportate.
    \[
        \resizebox{1\textwidth}{!}{$
            R_{sys} = R^8+8R^7 \cdot (1-R)+20R^6 \cdot (1-R)^2+16R^5 \cdot (1-R)^3+2 \cdot R^4(1-R)^4
        $}
    \]
Supponendo che la reliability di un singolo componente segua una legge esponenziale con $\lambda = 0.005$ e quindi la reliability di ogni nodo ha R=$e^{- \lambda t}$ posso calcolare la reliability per un mission time di 48 ore che sarà:
\begin{center}
    $
    R_{sys}(48) = 0.73
    $
\end{center}
Di seguito riportiamo anche il grafico della funzione di reliability del sistema
\begin{figure}[H]
    \centering
    \includegraphics[scale=0.6]{img/chap_5/plt_skip.png}
    \caption{Reliability skip-ring}
    \label{fig:plt_skip}
\end{figure}

\section{Esercizio 4}
\begin{figure}[H]
    \centering
    \includegraphics[scale=0.6]{img/chap_5/ex_4.png}
    \caption{Esercizio 4}
    \label{fig:plt_skip}
\end{figure}
\subsection{Confronto 1}
In termini di success path i due sistemi sono uguali (entrambe presentano 2 success path) ma eseguendo una analisi qualitativa se il sistema A fallisce nello schema di destra, l'intero sistema non funziona facendo dunque di A un single point of failure (SPOF).\\
Eseguendo i calcoli avremo:
\begin{center}
    $
    R_{sys1} = 1-(1-R_a \cdot R_b)(R_a \cdot R_c)
    R_{sys2} = R_a \cdot [1-(1-R_b)(1-R_b)]
    $
\end{center}
Ora possiamo particolarizzare per il valore di reliability che ci è stato fornito ricordando che MTTF = $\frac{1}{\lambda}$ e la reliability assume una occorrenza esponenziale, dunque $R=e^{-\lambda t}$.
Avendo fatto le seguenti considerazioni i valori del failure rate saranno:
\begin{itemize}
    \item $\lambda_a = 0.002$
    \item $\lambda_b = 0.00011$
    \item $\lambda_a = 0.0001$
\end{itemize}
Di seguito riportiamo la reliability in funzione del mission time.\\
\begin{figure}[H]
    \centering
    \includegraphics[scale=0.6]{img/chap_5/plt_confr_1.png}
    \caption{confronto tra i due sistemi}
    \label{fig:cofr_sys}
\end{figure}
Come si può vedere dal grafico \ref{fig:cofr_sys} il primo sistema è sempre più reliable del secondo.
\subsection{Confronto 2}
In questo confronto si può notare che il primo ha più success path del secondo e quindi in prima analisi si potrebbe pensare che il primo sia più reliable del secondo.\\
Però possiamo notare una prima cosa, i due sistemi presentano lo stesso SPOF che sarebbe A.\\
Procedendo con una analisi grafica avremo le seguenti espressioni per i due sistemi:
\begin{center}
    $
    R_{sys1} = R_a \cdot [1-(1-R_a)(1-R_b)]
    R_{sys2} = R_a
    $
\end{center}
Di seguito riportiamo il grafico della reliability dei due sistemi:
\begin{figure}[H]
    \centering
    \includegraphics[scale=0.6]{img/chap_5/plt_confr_2.png}
    \caption{confronto tra i due sistemi}
    \label{fig:cofr2_sys}
\end{figure}
\noindent
Come si può vedere dal grafico \ref{fig:cofr2_sys} il primo sistema è di poco più reliable del secondo con i parametri di MTTF forniti.\\
Quindi in termini di costo si potrebbe optare di usare più il sistema a destra rispetto che il sistema a sinistra che sicuramente costa di più.
\subsection{Confronto 3}
In questo caso abbiamo una situazione simile al secondo confronto poichè abbiamo due elementi serie comuni tra i due schemi.\\
Il primo schema ha 2 success path quindi è candidato ad essere più reliable di quello a destra, però ci interessa capire se questa differenza è molto ampia o no.\\
Di seguito riportiamo le espressioni dei due sistemi:
\begin{center}
    $
    R_{sys1} = R_a R_b \cdot [1-(1-R_a)(1-R_b)]
    R_{sys2} = R_a R_b
    $
\end{center}
Di seguito riportiamo il grafico della reliability dei due sistemi:
\begin{figure}[H]
    \centering
    \includegraphics[scale=0.6]{img/chap_5/plt_confr_3.png}
    \caption{confronto tra i due sistemi}
    \label{fig:cofr3_sys}
\end{figure}
\noindent
Anche in questo caso la differenza è poca quindi in termini di costo potrebbe essere più conveniente usare il sistema a destra poichè più economico sicuramente rispetto a quello di sinistra.
\subsection{Confronto 4}
Per quanto riguarda l'ultimo confronto si può vedere che il sistema di sinistra ha più success path di quello a destra e quindi possiano aspettarci una reliability più alta.\\
D'altronde l'osservazione è anche intuitiva poichè il sistema di sinistra introduce ridondanza mentre quello di destra no.\\
Procedendo in modo analitico avremo:
\begin{center}
    $
    R_{sys1} = 1-(1-R_a)(1-R_a R_b)
    R_{sys2} = R_a 
    $ 
\end{center}
Di seguito riportiamo il grafico della reliability dei due sistemi:
\begin{figure}[H]
    \centering
    \includegraphics[scale=0.5]{img/chap_5/plt_confr_4.png}
    \caption{confronto tra i due sistemi}
    \label{fig:cofr4_sys}
\end{figure}
\noindent
Come potevamo aspettarci la reliability del sistema a sinistra è molto migliore rispetto a quello a destra.
\section{Esercizio 5}

\begin{figure}[H]
    \centering
    \includegraphics[scale=0.5]{img/chap_5/ex_5.png}
    \caption{Esercizio 5}
    \label{fig:ex_5}
\end{figure}
\noindent
Del sistema in figura \ref{fig:ex_5} possiamo disegnare quello che è l'RBD caratterizzato dalla serie di tutti i componenti necessari al funzionamento del veivolo ed il parallelo delle repliche.\\
Lo schema risultante sarà quello in figura \ref{fig:rbd_5}
\begin{figure}[H]
    \centering
    \includegraphics[scale=0.5]{img/chap_5/rbd_5.png}
    \caption{RBD elicottero}
    \label{fig:rbd_5}
\end{figure}
\noindent
Di questo RBD possiamo calcolare la reliability complessiva del sistema:
\begin{center}
    $
    R_{sys} = [1-(1-R_p)^2] \cdot [1-(1-R_t)^2] \cdot [1-(1-R_{busA})^2] \cdot [1-(1-R_{busB})^2] \cdot [1-(1-R_{INS})(1-R_D)(1-(1-R_{AHRS})^3]
    $
\end{center}
Per effettuare l'FTA si è usato il tool \textbf{Top Event FTA} andando a modellare il sistema come un \textit{Fault Tree}.\\
Il Fault tree è ottenibile dall'RBD andando a sostituire a tutti i paralleli delle porte OR e a tutte le serie delle porte AND.\\
Facendo così otterremo il seguente schema:
\begin{figure}[H]
    \centering
    \includegraphics[scale=0.5]{img/chap_5/FTA_elicottero.png}
    \caption{FT elicottero}
    \label{fig:ft_elicottero}
\end{figure}
\noindent
Usando l'algoritmo MOCUS per individuare i \textit{Minimal Cut Set}, ovvero insiemi minimali di sottoinsiemi il cui malfunzionamento causa il fallimento del sistema.\\
In particolare quelli individuati sono i seguenti:
\begin{figure}[H]
    \centering
    \includegraphics[scale=0.4]{img/chap_5/minCutSet.png}
    \caption{Minimal Cut Set}
    \label{fig:cutSet}
\end{figure}
\noindent
Ora è richiesto di calcolare la reliability per 1 ora di volo dati i valori di MTTF riportati in tabella e supponendo una distribuzione esponenziale assumendo una coverage iniziale perfetta.\\
Secondo i valori riportati avremo:
\begin{center}
    $
    R_{sys}(t=1h) = 0.9999999400737104
    $
\end{center}
Per tale richiesta è stato anche disegnata la funzione di reliability del sistema.\\
\begin{figure}[H]
    \centering
    \includegraphics[scale=0.4]{img/chap_5/rel_graph.png}
    \caption{reliability elicottero}
    \label{fig:graph_rel}
\end{figure}
\noindent
Come ultima parte si richiede di calcolare la reliability per un ora di volo considerando un fattore di coverage per la \textit{fault detection and reconfiguration} della processing unit.\\
In particolare stiamo assumendo un processore \textit{spare} ovvero di ricambio, modifica la formula della reliability nel seguente modo:
\begin{center}
    $
    R_{sys} = [R_p+c(1-R_p)R_p] \cdot [1-(1-R_t)^2] \cdot [1-(1-R_{busA})^2] \cdot [1-(1-R_{busB})^2] \cdot [1-(1-R_{INS})(1-R_D)(1-(1-R_{AHRS})^3]
    $
\end{center}
Dobbiamo calcolare c tale che $R_{sys} \geq 0.99999$ per 1 ora di volo, risolvendo avremo almeno: 
\begin{center}
    $
        c \simeq 0.91
    $
\end{center}