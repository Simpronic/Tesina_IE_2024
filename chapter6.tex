\chapter{FFDA}
L'obiettivo della Field Failure Data Analysis(FFDA) è quello di ricavare delle misure di \textit{dependability} del sistema reale attraverso azioni di \textit{logging and collection, filtering and manipulation} e analisi di failure che si verificano durante la normale fase operativa del sistema.\\
Questa tecnica measurement based è estremamente affidabile e consente un analisi dettagliata e precisa delle proprietà di un sistema, ma è molto costosa e difficilmente generalizzabile.\\
In questo capitolo si descrive lo svolgimento di una \textbf{log-based FFDA} condotta su due sistemi:
\begin{itemize}
\item \textbf{Mercury}: situato presso il NCSA
\item \textbf{BlueGene}: situato presso LLNL
\end{itemize}
\section{Analisi sistema Mecury}
Il supercalcolatore Mercury è costituito da nodi IBM attraverso una architettura 3-layered + nodo di management.\\
Su ogni nodo è installato un sistema \textbf{RedHat 9.0 OS} e i log sono stati raccolti attraverso \textbf{syslog deamon}.\\
Si hanno a disposizione dei log già filtrati e deperametrizzati tramite whitelisting con dizionario che ammontano ad 80854 di severity \textbf{FATAL}.\\
Ogni entry ha un formato pari a:
\begin{center}
timestamp - nodo - sottosistema originante - descrizione
\end{center}
Riportiamo di seguito le categorie di errore che ritroviamo nei log:
\begin{itemize}
\item \textbf{DEV}:  Platform PCI Component Error Info Section;
\item \textbf{MEM}: MEM: Mem Error Detail: Physical Address *, Address Mask:*, Node:*, Card:*, Module: *, Bank:*, Device:*,Row:*,Column:*
\item \textbf{NET}: Connection down
\item \textbf{I-O}:  error, dev *:* (hda), sector *; hda: packet command error:
\item \textbf{PRO}: BEGIN HARDWARE ERROR STATE AT CMC. 
\end{itemize}
In generale abbiamo 50 messaggi di errore univoci che hanno generato 80854 entry.
\subsection{Analisi di alto livello del sistema}
La prima analisi che si è svolta è una analisi di alto livello riguardo gli errori del sistema, quindi quale è il più frequente come errore e quale nodo ha la più alta frequenza di fault.\\
In particolare si può notare, dalla figura \ref{fig:freqFaultNode} il nodo tg-c401 è quello che ha associato il più alto numero di errori con circa il 77\% di essi associati a questo nodo 
\begin{figure}[H]
    \centering
    \includegraphics[scale=0.4]{img/chap_6/FaultFreqNode.png}
    \caption{Frequenza di faults per nodo}
    \label{fig:freqFaultNode}
\end{figure}
\noindent
Inoltre si è osservato anche, secondo il grafico \ref{fig:freqFauls} che la categoria più frequente è quella DEV con il 71\% di occorrenze.
\begin{figure}[H]
    \centering
    \includegraphics[scale=0.4]{img/chap_6/FaultFreq.png}
    \caption{Frequenza di tipi faults}
    \label{fig:freqFauls}
\end{figure}
\noindent
Analizzando il grafico \ref{fig:freqFaultNode} si è deciso di analizzare il breakup delle entry per nodo e categoria solo per i primi 8 nodi presenti nel grafico.\\
Si può notare dal grafico \ref{fig:freqFaulsTypeNode} che i nodi provenienti dallo stello layer esibiscono simili comportamenti di failure.
\begin{figure}[H]
    \centering
    \includegraphics[scale=0.3]{img/chap_6/FaultTypeNodeFreq.png}
    \caption{Frequenza di tipi faults per nodo}
    \label{fig:freqFaulsTypeNode}
\end{figure}
\noindent
Ad esempio i nodi computazionali tendono ad avere errori di tipo DEV e MEM, mentre i nodi di storage esibiscono errori di tipo I/O.\\
Il nodo \textit{tg-master} è caratterizzato dall'occorrenza NET error probabilmente causati da errori di comunicazione con altri nodi.\\
\subsection{Coalescenze window and content-based coalescence technique}
L'analisi di coalescenza è una fase molto critica nell'analisi dei log poichè ha lo scopo di ricostruire il processo di fallimento del sistema raggruppando gli eventi scaturiti dalla stessa failure.\\
Per effettuare tale analisi dobbiamo determinare una \textit{coalescence window}(cwin) per applicare una coalescenza temporale (\textbf{toupling}).\\
Per determinare una cwin adeguata si è svolta una analisi di sensibilità sul numero di tuple generate in funzione della dimensione della finestra scelta.
 \begin{figure}[H]
    \centering
    \includegraphics[scale=0.4]{img/chap_6/cwin.png}
    \caption{Cwin}
    \label{fig:cwindow}
\end{figure}
\noindent
Una finestra di coalescenza troppo piccola potrebbe produrre un fenomeno di \textbf{troncamento} (ovvero tuple contigue potrebbero contenere log associati alla stessa failure), una troppo grande invece provoca il fenomeno di \textbf{collissione} (una tupla contiene log appartenenti a diverse failure indipendenti), inducendo analisi di reliability pessimistiche o ottimistiche.\\
In particolare una collisione potrebbe portare ad una analisi di reliability ottimistica, e dunque pericolosa, mentre dei troncamenti possono portare ad analisi di reliability pessimistiche.\\
Per la scelta della cwin si è deciso di usare la regola empirica di scegliere un punto del ginocchio della curva riportata in figura \ref{fig:cwindow} che in tale caso è stato scelto come 160s, generando 594 tuple.\\
Prima di proseguire però per la reliability empirica è stata svolta una analisi di bontà della scelta della coalescence window andando a valutare quelli che sono i presunti troncamenti e collisioni nelle tuple ottenute.\\
\section{BlueGene}
\subsection{Overview}
Blue Gene è un tipo di architettura con l'obbiettivo di realizzare dei supercomputer con potenze di calcolo che vanno dalle decine di teraFLOPS per arrivare fino ai petaFLOPS con un consumo energetico relativamente basso.
 \begin{figure}[H]
	\centering
	\includegraphics[scale=0.4]{img/chap_6/BlueGene1.png}
	\caption{BlueGene Sample diagram}
	\label{fig:BlueGeneSamplediagram}
\end{figure}
\noindent
Si hanno una serie di nodi disposti in 64 rack e catalogati in base a \textbf{midplane}, \textbf{compute card} e \textbf{compute chip}
Per quanto riguarda i dati a disposizione dei file log, si hanno a disposizione 125624 fatal log entry.
Il formato dei log è confrontabile con quello dei log del sistema analizzato precedentemente.
Si hanno infatti time-stamp, originating node, originating card, test-free message.
Si riporti brevemente, prima di procedere con analisi successive, le distribuzioni di  failure registrate e riportate nel log file nel tempo.
 \begin{figure}[H]
	\centering
	\includegraphics[scale=0.4]{img/chap_6/BlueGene2.png}
	\caption{BlueGene failure distribution}
	\label{fig:Bg failure distr}
\end{figure}
 \begin{figure}[H]
	\centering
	\includegraphics[scale=0.4]{img/chap_6/BlueGene3.png}
	\caption{BlueGene failure entry distribution per singolo nodo}
	\label{fig:Bg failure distr sing nod}
\end{figure}
 \begin{figure}[H]
	\centering
	\includegraphics[scale=0.4]{img/chap_6/BlueGene4.png}
	\caption{BlueGene failure entry distribution per singola computation card}
	\label{fig:Bg failure distr sing comp card}
\end{figure}
 \begin{figure}[H]
	\centering
	\includegraphics[scale=0.4]{img/chap_6/BlueGene5.png}
	\caption{BlueGene failure entry distribution per singola computation chip}
	\label{fig:Bg failure distr sing comp chip}
\end{figure}
\noindent
E' possibile osservare da questi grafici che, a differenza del sistema precedente, le entry failure sono più omogenee nel tempo, non essendoci dei picchi di failure per un tempo specifico di periodi operativi del sistema.
Dal grafico dei nodi \ref{fig:Bg failure distr sing nod} si nota che non ci sono nodi del sistema che causano una percentuale significativa dei guasti.
Infatti, le failure sono generate dai 700 nodi presenti nel file log preso in considerazione.
Nella figura successiva, \ref{fig:Bg failure distr sing comp card} è possibile vedere le entry per ogni card. 
Questa distribuzione mostra come i fallimenti maggiori sono relativi alle card in posizione multipla di quattro. Infatti, cumulativamente raggiungono circa l'85\% del totale dei fallimenti
Infine, nell'ultima figura \ref{fig:Bg failure distr sing comp chip} si riporta il grafico relativo alle failure in funzione dei chip.
Qui è possibile visionare invece, a differenza dei nodi, che per i chip \textit{J18-U11} e \textit{J18-U01} è presente la maggior parte  dei failure, a tal punto che cumulativamente, entrambi i chip raggiungono quasi l'80\% di tutte le failure presenti.