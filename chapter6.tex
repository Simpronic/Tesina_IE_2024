
\chapter{FFDA}
L'obiettivo della Field Failure Data Analysis(FFDA) è quello di ricavare delle misure di \textit{dependability} del sistema reale attraverso azioni di \textit{logging and collection, filtering and manipulation} e analisi di failure che si verificano durante la normale fase operativa del sistema.\\
Questa tecnica measurement based è estremamente affidabile e consente un analisi dettagliata e precisa delle proprietà di un sistema, ma è molto costosa e difficilmente generalizzabile.\\
In questo capitolo si descrive lo svolgimento di una \textbf{log-based FFDA} condotta su due sistemi:
\begin{itemize}
\item \textbf{Mercury}: situato presso il NCSA
\item \textbf{BlueGene}: situato presso LLNL
\end{itemize}
\section{Analisi sistema Mecury}
Il supercalcolatore Mercury è costituito da nodi IBM attraverso una architettura 3-layered + nodo di management.\\
Su ogni nodo è installato un sistema \textbf{RedHat 9.0 OS} e i log sono stati raccolti attraverso \textbf{syslog deamon}.\\
Si hanno a disposizione dei log già filtrati e deperametrizzati tramite whitelisting con dizionario che ammontano ad 80854 di severity \textbf{FATAL}.\\
Ogni entry ha un formato pari a:
\begin{center}
timestamp - nodo - sottosistema originante - descrizione
\end{center}
Riportiamo di seguito le categorie di errore che ritroviamo nei log:
\begin{itemize}
\item \textbf{DEV}:  Platform PCI Component Error Info Section;
\item \textbf{MEM}: MEM: Mem Error Detail: Physical Address *, Address Mask:*, Node:*, Card:*, Module: *, Bank:*, Device:*,Row:*,Column:*
\item \textbf{NET}: Connection down
\item \textbf{I-O}:  error, dev *:* (hda), sector *; hda: packet command error:
\item \textbf{PRO}: BEGIN HARDWARE ERROR STATE AT CMC. 
\end{itemize}
In generale abbiamo 50 messaggi di errore univoci che hanno generato 80854 entry.
\subsection{Analisi di alto livello del sistema}
La prima analisi che si è svolta è una analisi di alto livello riguardo gli errori del sistema, quindi quale è il più frequente come errore e quale nodo ha la più alta frequenza di fault.\\
In particolare si può notare, dalla figura \ref{fig:freqFaultNode} il nodo tg-c401 è quello che ha associato il più alto numero di errori con circa il 77\% di essi associati a questo nodo 
\begin{figure}[H]
    \centering
    \includegraphics[scale=0.4]{img/chap_6/FaultFreqNode.png}
    \caption{Frequenza di faults per nodo}
    \label{fig:freqFaultNode}
\end{figure}
\noindent
Inoltre si è osservato anche, secondo il grafico \ref{fig:freqFauls} che la categoria più frequente è quella DEV con il 71\% di occorrenze.
\begin{figure}[H]
    \centering
    \includegraphics[scale=0.4]{img/chap_6/FaultFreq.png}
    \caption{Frequenza di tipi faults}
    \label{fig:freqFauls}
\end{figure}
\noindent
Analizzando il grafico \ref{fig:freqFaultNode} si è deciso di analizzare il breakup delle entry per nodo e categoria solo per i primi 8 nodi presenti nel grafico.\\
Si può notare dal grafico \ref{fig:freqFaulsTypeNode} che i nodi provenienti dallo stello layer esibiscono simili comportamenti di failure.
\begin{figure}[H]
    \centering
    \includegraphics[scale=0.3]{img/chap_6/FaultTypeNodeFreq.png}
    \caption{Frequenza di tipi faults per nodo}
    \label{fig:freqFaulsTypeNode}
\end{figure}
\noindent
Ad esempio i nodi computazionali tendono ad avere errori di tipo DEV e MEM, mentre i nodi di storage esibiscono errori di tipo I/O.\\
Il nodo \textit{tg-master} è caratterizzato dall'occorrenza NET error probabilmente causati da errori di comunicazione con altri nodi.\\
\subsection{Coalescenze window and content-based coalescence technique}
L'analisi di coalescenza è una fase molto critica nell'analisi dei log poichè ha lo scopo di ricostruire il processo di fallimento del sistema raggruppando gli eventi scaturiti dalla stessa failure.\\
Per effettuare tale analisi dobbiamo determinare una \textit{coalescence window}(cwin) per applicare una coalescenza temporale (\textbf{toupling}).\\
Per determinare una cwin adeguata si è svolta una analisi di sensibilità sul numero di tuple generate in funzione della dimensione della finestra scelta.
 \begin{figure}[H]
    \centering
    \includegraphics[scale=0.4]{img/chap_6/cwin.png}
    \caption{Cwin}
    \label{fig:cwindow}
\end{figure}
\noindent
Una finestra di coalescenza troppo piccola potrebbe produrre un fenomeno di \textbf{troncamento} (ovvero tuple contigue potrebbero contenere log associati alla stessa failure), una troppo grande invece provoca il fenomeno di \textbf{collissione} (una tupla contiene log appartenenti a diverse failure indipendenti), inducendo analisi di reliability pessimistiche o ottimistiche.\\
In particolare una collisione potrebbe portare ad una analisi di reliability ottimistica, e dunque pericolosa, mentre dei troncamenti possono portare ad analisi di reliability pessimistiche.\\
Per la scelta della cwin si è deciso di usare la regola empirica di scegliere un punto del ginocchio della curva riportata in figura \ref{fig:cwindow} che in tale caso è stato scelto come 160s, generando 594 tuple.\\
Prima di proseguire però per la reliability empirica è stata svolta una analisi di bontà della scelta della coalescence window andando a valutare quelli che sono i presunti troncamenti e collisioni nelle tuple ottenute.\\
 \begin{figure}[H]
    \centering
    \includegraphics[scale=0.4]{img/chap_6/Analisi_troncamenti_e_collisioni.png}
    \caption{Estratto da Analisi troncamenti e collisioni}
    \label{fig:AnalTroncCollision}
\end{figure}
\noindent
Le statistiche riportate in figura (che sono però un estratto) hanno permesso di calcolare una percentuale di tuple con troncamento rispetto a quelle totali.\\
Si è supposto di essere in presenza di troncamenti al verificarsi delle seguenti condizioni:
\begin{itemize}
\item Valore di interarrivo prossimo al valore della cwin 
\item Tuple caratterizzati da nodi contigui con fallimenti degli stessi sottosistemi
\end{itemize}
Con le condizioni sopra citate abbiamo una percentuale del 17.2\% di troncamenti.\\
Per migliorare tale percentuale si è deciso di effettuare una strategia di coalescenza non sono di tipo \textbf{time} ma anche \textbf{content-based}, ovvero una entry viene inserita in una tupla se rispetta il vincolo della finsetra o, se non rispetta il vincolo della finestra , rispetta il vincolo della finestra doppia e il nodo ed il componente interessato sono gli stessi dell'ultimo log inserito nella tupla.\\
Ripetendo il tupling con la precedente osservazione avrò 528 tuple con un tasso di troncamento del 10\% usando sempre una cwin di 160s.\\
 \begin{figure}[H]
    \centering
    \includegraphics[scale=0.4]{img/chap_6/Analisi_troncamenti_e_collisioni_2.png}
    \caption{Estratto da Analisi troncamenti e collisioni 2}
    \label{fig:AnalTroncCollision2}
\end{figure}
\noindent
\subsection{Reliability model}
Definito il valore di coalescence window che è stato ritenuto adeguato. sfruttando i tempi di interarrivo delle tuple, si può realizzare un grafico empirico della reliability del modello nel suo complesso.\\
 \begin{figure}[H]
    \centering
    \includegraphics[scale=0.4]{img/chap_6/EmpRel_Mercuty.png}
    \caption{Reliability empirica}
    \label{fig:empRel}
\end{figure}
\noindent
Il passo successivo è quello di inferire un modello di reliability regressivo (non lineare) dai dati sperimentali che è solo ai fini di fitting.\\
Di seguito riportiamo i modelli inferiti con i loro parametri.\\
 \begin{figure}[H]
    \centering
    \includegraphics[scale=0.4]{img/chap_6/rel_exp.png}
     \includegraphics[scale=0.4]{img/chap_6/res_exp.png}
    \caption{Fitting esponenziale}
    \label{fig:fitexp}
\end{figure}
 \begin{figure}[H]
    \centering
    \includegraphics[scale=0.4]{img/chap_6/rel_hyp_exp.png}
     \includegraphics[scale=0.4]{img/chap_6/res_hyp_exp.png}
    \caption{Fitting iper-esponenziale}
    \label{fig:fithypexp}
\end{figure}
 \begin{figure}[H]
    \centering
    \includegraphics[scale=0.4]{img/chap_6/fit_stats.png}
    \caption{Statistiche}
    \label{fig:stats}
\end{figure}
\noindent
Come si può leggere dalle figure sono stati scelti due fitting che sono molto accurati rispetto ai dati in nostro possesso.\\
Di seguito riportiamo i due tipi di fitting:
\begin{center}
$exp(t) = ae^{bt}$ \\
$hypExp(t) = ae^{bt} + ce^{dt}$
\end{center}
Come è osservabile dalla tabella \ref{fig:stats} il modello più preciso è quello iper-esponenziale per descrivere l'andamento della reliability del sistema.\\
Per verificare la significatività del modello si è poi applicato un test non parametrico di tipo Kolmogorov-Smirnov, ottenendo i seguenti risultati.\\
\begin{table}[htbp]
    \centering
    \label{tab:esempio}
    \begin{tabular}{|c|c|c|} % specifica il numero e l'allineamento delle colonne (c = centrato, l = sinistra, r = destra)
        \hline
        Modello & $R^2$ & $p-value$ \\ % separa le celle con '&', e termina ogni riga con '\\'
        \hline
        Esponenziale & 0.947 &  7.4306e-06\\
        Iper-esponenziale & 0.995 &  0.6872\\
        \hline
    \end{tabular}
\end{table}\\
\noindent
Dalla precedente possiamo notare che il test rigetta l'ipotesi nulla con una confidenza del 95\% per il modello esponenziale, mentre invece non rigetta l'ipotesi nulla per un modello iper-esponenziale.\\
Si può infine calcolare l'MTTF del sistema usando l'integrazione, ovvero:
\begin{center}
$MTTF_{sys} = \int_{0}^{\infty} R(t) dt = 1.0271s$
\end{center}
\subsection{Analisi dei nodi}
Definita la misura di reliability del sistema si è pensato di analizzare i singoli nodi andando a valutare il numero di failure per nodo.\\
Per fare ciò si è filtrato il file di log in base ai nodi in esame che sono stati i primi 8 che davano più contributo alle failure.\\
Dopo aver filtrato si è proceduto ad effettuare una analisi per la cwin per ogni sistema che riportiamo di seguito.\\
\begin{figure}[H]
    \centering
    \begin{subfigure}{0.25\textwidth}
        \centering
        \includegraphics[scale=0.2]{img/chap_6/tg-c117_cwin.png}
        \caption{tg-c117}
        \label{fig:tgc117}
    \end{subfigure}
    \begin{subfigure}{0.25\textwidth}
        \centering
        \includegraphics[scale=0.2]{img/chap_6/tg-c238_cwin.png}
        \caption{tg-c238}
        \label{fig:tgc238}
    \end{subfigure}
\end{figure}
\begin{figure}[h]
    \centering
    \begin{subfigure}{0.25\textwidth}
        \centering
        \includegraphics[scale=0.2]{img/chap_6/tg-c242_cwin.png}
        \caption{tg-c242}
        \label{fig:tgc242}
    \end{subfigure}
    \begin{subfigure}{0.25\textwidth}
        \centering
        \includegraphics[scale=0.2]{img/chap_6/tg-c401_cwin.png}
        \caption{tg-c401}
        \label{fig:tgc401}
    \end{subfigure}
\end{figure}
\begin{figure}[H]
    \centering
    \begin{subfigure}{0.25\textwidth}
        \centering
        \includegraphics[scale=0.2]{img/chap_6/tg-c572_cwin.png}
        \caption{tg-c572}
        \label{fig:tgc572}
    \end{subfigure}
    \begin{subfigure}{0.25\textwidth}
        \centering
        \includegraphics[scale=0.2]{img/chap_6/tg-c648_cwin.png}
        \caption{tg-c648}
        \label{fig:tgc648}
    \end{subfigure}
\end{figure}
\begin{figure}[H]
    \centering
     \begin{subfigure}{0.25\textwidth}
        \centering
        \includegraphics[scale=0.2]{img/chap_6/tg-c669_cwin.png}
        \caption{tg-c669}
        \label{fig:tgc669}
    \end{subfigure}
    \begin{subfigure}{0.25\textwidth}
        \centering
        \includegraphics[scale=0.2]{img/chap_6/tg-login3_cwin.png}
        \caption{tg-login3}
        \label{fig:tglogin3}
    \end{subfigure}
\end{figure}
\noindent
Per ogni modello è stato scelto un calore di coalescence window in corrispondenza della curva corrispondente.\\
Dunque avremo la seguente tabella:
 \begin{table}[htbp]
    \centering
    \label{tab:esempio}
    \begin{tabular}{|c|c|} % specifica il numero e l'allineamento delle colonne (c = centrato, l = sinistra, r = destra)
        \hline
        Node & Cwin  \\ % separa le celle con '&', e termina ogni riga con '\\'
        \hline
        tg-c117 & 800 \\
        tg-c238 & 120 \\
        tg-c242 & 210 \\
        tg-c401 & 230 \\
        tg-c572 & 800 \\
        tg-c648 & 800 \\
        tg-c669 & 150 \\
        tg-login3 & 190\\
        sistema & 160\\
        \hline
    \end{tabular}
\end{table}
\noindent
Come si può vedere dalla tabella solo il nodo tg-c401 è confrontabile con la finestra del sistema poichè è quella più influente per quest'ultimo.\\
Dopo questa analisi si è proceduto alla derivazione della reliability empirica per ogni nodo e ne sono stati stimati i parametri, di seguito si riportano i parametri ricavati tramite curve fitter.\\
\begin{figure}[H]
    \centering
    \includegraphics[scale=0.4]{img/chap_6/EmpRelStats.png}
    \caption{Statistiche di fitting per la reliabilty}
    \label{fig:StatsempRel}
\end{figure}
\noindent
Per i nodi tg-c572, tg-c648 e tg-login3 non è stato possibile calcolare la reliability empirica poichè provvisti di poche tuple.\\
Di seguito riportiamo una tabella riassuntiva dei risultati del test non parametrico di Kolmogorov-Smirnov.\\
 \begin{table}[htbp]
    \centering
    \label{tab:esempio}
    \begin{tabular}{|c|c|c|} % specifica il numero e l'allineamento delle colonne (c = centrato, l = sinistra, r = destra)
        \hline
        Node & Modello &p-value \\ % separa le celle con '&', e termina ogni riga con '\\'
        \hline
        tg-c117 & iper-esponenziale & 0.995 \\
        tg-c238 & iper-esponenziale & 0.999 \\
        tg-c242 & iper-esponenziale & 0.993 \\
        tg-c401 & iper-esponenziale & 0.6759 \\
        tg-c669 & esponenziale & 0.9950 \\
        \hline
    \end{tabular}
\end{table}\\
In ultima analisi si è deciso di confrontare le varie reliability dei componenti in modo da identificare un eventuale reliability bottleneck.\\
 \begin{figure}[H]
	\centering
	\includegraphics[scale=0.3]{img/chap_6/RelComp.png}
	\caption{Grafico comparativo reliability}
	\label{fig:SuntoReliab}
\end{figure}
\noindent
Come apprezzabile dal grafico \ref{fig:SuntoReliab} ovviamente il sistema ha la reliability più bassa, ma il vero bottleneck della reliability è il nodo tg-c401 che, in rapporto con le sue entry presenta una reliability minore rispetto agli altri.\\
Dal grafico può risultare il nodo tg-c242 ad essere il bottleneck ma per tale nodo disponiamo di molti meno elementi del tg-c401 per cui abbiamo decidere di considerare quest'ultimo il vero bottleneck.
\section{BlueGene}
\subsection{Overview}
Blue Gene è un tipo di architettura con l'obbiettivo di realizzare dei supercomputer con potenze di calcolo che vanno dalle decine di teraFLOPS per arrivare fino ai petaFLOPS con un consumo energetico relativamente basso.
 \begin{figure}[H]
	\centering
	\includegraphics[scale=0.4]{img/chap_6/BlueGene1.png}
	\caption{BlueGene Sample diagram}
	\label{fig:BlueGeneSamplediagram}
\end{figure}
\noindent
Si hanno una serie di nodi disposti in 64 rack e catalogati in base a \textbf{midplane}, \textbf{compute card} e \textbf{compute chip}
Per quanto riguarda i dati a disposizione dei file log, si hanno a disposizione 125624 fatal log entry.
Il formato dei log è confrontabile con quello dei log del sistema analizzato precedentemente.
Si hanno infatti time-stamp, originating node, originating card, test-free message.
Si riporti brevemente, prima di procedere con analisi successive, le distribuzioni di  failure registrate e riportate nel log file nel tempo.
 \begin{figure}[H]
	\centering
	\includegraphics[scale=0.4]{img/chap_6/BlueGene2.png}
	\caption{BlueGene failure distribution}
	\label{fig:Bg failure distr}
\end{figure}
 \begin{figure}[H]
	\centering
	\includegraphics[scale=0.4]{img/chap_6/BlueGene3.png}
	\caption{BlueGene failure entry distribution per singolo nodo}
	\label{fig:Bg failure distr sing nod}
\end{figure}
 \begin{figure}[H]
	\centering
	\includegraphics[scale=0.4]{img/chap_6/BlueGene4.png}
	\caption{BlueGene failure entry distribution per singola computation card}
	\label{fig:Bg failure distr sing comp card}
\end{figure}
 \begin{figure}[H]
	\centering
	\includegraphics[scale=0.4]{img/chap_6/BlueGene5.png}
	\caption{BlueGene failure entry distribution per singola computation chip}
	\label{fig:Bg failure distr sing comp chip}
\end{figure}
\noindent
E' possibile osservare da questi grafici che, a differenza del sistema precedente, le entry failure sono più omogenee nel tempo, non essendoci dei picchi di failure per un tempo specifico di periodi operativi del sistema.
Dal grafico dei nodi \ref{fig:Bg failure distr sing nod} si nota che non ci sono nodi del sistema che causano una percentuale significativa dei guasti.
Infatti, le failure sono generate dai 700 nodi presenti nel file log preso in considerazione.
Nella figura successiva, \ref{fig:Bg failure distr sing comp card} è possibile vedere le entry per ogni card. 
Questa distribuzione mostra come i fallimenti maggiori sono relativi alle card in posizione multipla di quattro. Infatti, cumulativamente raggiungono circa l'85\% del totale dei fallimenti
Infine, nell'ultima figura \ref{fig:Bg failure distr sing comp chip} si riporta il grafico relativo alle failure in funzione dei chip.
Qui è possibile visionare invece, a differenza dei nodi, che per i chip \textit{J18-U11} e \textit{J18-U01} è presente la maggior parte  dei failure, a tal punto che cumulativamente, entrambi i chip raggiungono quasi l'80\% di tutte le failure presenti.
\subsection{Coalescence window}
Per la coalescence window e per la sua size e per la scelta dell'euristica del tuplig, sono stati eseguiti gli stessi criteri fatti per il sistema Mercury.
 \begin{figure}[H]
	\centering
	\includegraphics[scale=0.4]{img/chap_6/BlueGene6.png}
	\caption{BlueGene failure analisi di sensitività}
	\label{fig:BlueGene failure analisi di sensitività}
\end{figure}
\noindent
Il raggruppamento delle tuple è stato effettuato tramite lo script basato sull'approccio time content based e come window size è stato scelto un tempo t pari a 230s, ottenendo 396 tuple.
\subsection{Reliability model}
Con i tempi di interarrivo delle tuple è stato possibile costruire un grafico di reliability empirico per il sistema in esame
\begin{figure}[H]
	\centering
	\includegraphics[scale=0.4]{img/chap_6/BlueGene7.png}
	\caption{BlueGene reliability modello empirico}
	\label{fig:BlueGene reliability}
\end{figure}
\noindent
Successivamente, tramite l'utilizzo del tool curve fitter, è stato possibile costruire un modello di reliability regressivo dai dati sperimentali.
\begin{figure}[H]
	\centering
	\includegraphics[scale=0.4]{img/chap_6/BlueGene8.png}
	\caption{BlueGene reliability modello di reliability regressivo esponenziale}
	\label{fig:BlueGene reliability esponenziale}
\end{figure}
\begin{figure}[H]
	\centering
	\includegraphics[scale=0.4]{img/chap_6/BlueGene9.png}
	\caption{BlueGene reliability modello di reliability regressivo iper-esponenziale}
	\label{fig:BlueGene reliability iper-esponenziale}
\end{figure}
\noindent
Di seguito invece la tabella relativa ai due grafici riportati 
\begin{figure}[H]
	\centering
	\includegraphics[scale=0.4]{img/chap_6/BlueGene10.png}
	\caption{BlueGene reliability Tabella dei modelli}
	\label{fig:BlueGene reliability tabella}
\end{figure}
\noindent
\begin{figure}[H]
	\centering
	\includegraphics[scale=0.4]{img/chap_6/BlueGene11.png}
	\caption{BlueGene reliability Tabella dei modelli - exp}
	\label{fig:BlueGene reliability tabella exp}
\end{figure}
\begin{figure}[H]
	\centering
	\includegraphics[scale=0.4]{img/chap_6/BlueGene12.png}
	\caption{BlueGene reliability Tabella dei modelli - hyp\_exp }
	\label{fig:BlueGene reliability tabella hexp}
\end{figure}
\noindent
Come si può osservare dal valore $R^2$  restituitoci dal modello regressivo, il modello iper esponenziale descrive in maniera più efficiente i dati empirici.
Il test di \textit{Kolmogorov-Smirnov} fallisce nel rigettare l'ipotesi nulla in questo caso, confermando la significatività del modello con un p-value dello 0,89\%.
Il MTTF del sistema sarà:
$\\$
$MTTF_{sys} = \int_{0}^{\infty} R(t) dt = 18.252s$
\section{Confronto sistemi}
Dalle analisi svolte e dai valori di MTTF calcolati, all'infinito il sistema Blue Gene/L risulta essere più reliable di Mercury, però si è deciso comunque di fare una analisi grafica per individuare se eventualmente per un certo mission time non fosse valida tale ipotesi.\\
\begin{figure}[H]
	\centering
	\includegraphics[scale=0.25]{img/chap_6/SystemTest.jpeg}
	\caption{Confronto sistemi}
	\label{fig:SystemTesing}
\end{figure}
\noindent
Dal grafico riportato si può notare che in definitiva, per le reliability empiriche calcolate il sistema Blue Gene/L è più reliable per ogni valore di \textit{mission time}