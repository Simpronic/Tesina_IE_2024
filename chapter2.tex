\chapter{PCA \& Clustering}
In questo capitolo dell'elaborato effettueremo delle analisi di un workload utilizzando tecniche di analisi dati, ovvero PCA e clustering.
In breve la PCA è una procedura statistica che ci permette di riassumere l'informazione contenuta in una grande tabella attraverso un insieme ristretto di indici sintetici facilmente visualizzabili.
Invece il clustering ci permette di dividere dati non etichettati in insiemi differenti detti clusters attraverso una funzione di distanza.
\section{Traccia}
Dato un Workload reale, trovarne uno sintetico rispettando i seguenti requisiti: 

\begin{itemize}
    \item Trovare il miglior trade-off tra numero di componenti principali e numero di cluster
    \item Effettuare una analisi di sensitività, osservando come la varianza cambia al variare di componenti principali e numero di cluster
\end{itemize}
\section{Filtering}
Il primo passo effettuato è stato analizzare i dati grezzi su JMP per capire se si poteva effettuare un filtraggio.\\
Questa analisi hs rivelato, come indica la figura \ref{fig:analisi_filtering} la presenza di colonne a varianza nulla.
\begin{figure}[H]
    \centering
    \includegraphics[scale=0.6]{img/chap_2/PCA_CLUSTERING_1.PNG}
    \caption{Analisi colonne}
    \label{fig:analisi_filtering}
\end{figure}
\noindent
Per tale ragione, non essendo queste colonne portatrici di informazione, sono state eliminate dall'analisi.\\
\textbf{Osservazione:} Quando si parla di eliminare una colonna si intende l'intenzione di non considerarle all'interno dell'analisi ma in fase di selezione del workoad queste ultime devono essere incluse.
Dopo aver fatto una analisi sulle colonne, è stata eseguita una analisi anche riguardante le righe. 
Attraverso l'istogramma frequentista della figura \ref{fig:AnalisiRighe1} sono stati rilevati i seguenti outlier:
\begin{itemize}
    \item \textbf{Riga 1}: con riferimento alla figura \ref{fig:AnalisiRighe1} outlier corrispondente a tutte le grandezze indicate, probabilmente legato ad una condizione di basso carico
    \item \textbf{Righe 2-87}: con riferimento alla figura \ref{fig:AnalisiRighe2} outliers relativi ad una serie di parametri come la memoria (Mem-Free e Writeback). Tali outliers sono legati probabilmente ad una fase transitoria del sistema come l'accensione.
\end{itemize}
\begin{figure}[H]
    \centering
    \includegraphics[scale=0.4]{img/chap_2/Outlier_PCA.PNG}
    \caption{Istogramma frequentista 1}
    \label{fig:AnalisiRighe1}
\end{figure}
\begin{figure}[H]
    \centering
    \includegraphics[scale=0.5]{img/chap_2/righe_escluse.PNG}
    \caption{Istogramma frequentista 2}
    \label{fig:AnalisiRighe2}
\end{figure}
\noindent
Le righe sopracitate sono state eliminate dall'analisi.\\
Da questa procedura di filtering otteniamo un dataset filtrato sul quale effettuare delle analisi.
\section{PCA}
Come detto nell'introduzione, la PCA è una tecnica che effettua una trasformazione generando quelle che si chiamano componenti principali che sono in numero pari alle componenti sulle quali abbiamo effettuato la trasfomazione.\\
Poichè la PCA tiene conto della varianza ci permette di selezionare alcune di queste componenti per ridurre la dimensionalità.\\
Tale analisi viene svolta tramite l'uso di JMP e fornisce i seguenti output: 
\begin{itemize}
    \item \textbf{Loading Plot:} Mostra una matrice  di rappresentazione bidimensionale dei fattori di carico. Tale grafico ci permette di osservare che se due variabili sono simili, sono rappresentate molto vicine tra loro.
    \item \textbf{Score Plot:} mostra una matrice di dispersione per coppie di componenti principali, dunque spiega come si distribuiscono gli elementi in base alle due componenti selezionate.
    \item \textbf{Diagramma a barre:} mostra la varianza conservata da ogni componente principale.
\end{itemize}
Di seguito riportiamo tali plot:\\
\begin{figure}[H]
    \centering
    \includegraphics[scale=0.6]{img/chap_2/PCA_CLUSTERING_2.PNG}
    \caption{output PCA}
    \label{fig:Plot_PCA}
\end{figure}
\noindent
Analizzando più da vicino gli autovalori avremo la seguente situazione:
\begin{figure}[H]
    \centering
    \includegraphics[scale=0.6]{img/chap_2/Osservazione_PCA_Clustering.PNG}
    \caption{Autovalori con varianza}
    \label{fig:Osservazione}
\end{figure}
\noindent
Possiamo notare che gli autovalori dal 18 al 20 può sembrare non spieghino varianza, in realtà aumentando i decimali relativi alla percentuale cumulativa possiamo renderci conto che spiegano varianza, ma è molto piccola.\\
Tale osservazione è visibile nella figura \ref{fig:Osservazione}
Da questa figura ci rendiamo conto che solo il 20 difatto non spiega varianza.
Inoltre abbiamo fatto PCA su 21 dimensioni ma ci ritroviamo solo 20 autovalori.\\
Questo è dovuto ad una correlazione tra \textbf{MemFree} e \textbf{WriteBack}, tale correlazione è evidenziata dalla matrice di correlazione di figura \ref{fig:MatriceDiOsservazione}
\begin{figure}[H]
    \centering
    \includegraphics[scale=0.4]{img/chap_2/PCA_CLUSTERING_4.PNG}
    \caption{Matrice di correlazione}
    \label{fig:MatriceDiOsservazione}
\end{figure}
\section{Clustering}
Dopo aver effettuato l'estrazione delle componenti principali, il prossimo passo da effettuare è quello dell'operazione di clustering. In particolare, il clustering scelto è quello di tipo gerarchico utilizzando come criterio il Metodo di Ward, in modo tale da formare "cluster" in cui le somiglianze tra gli elementi siamo massime e quindi minimizzare la varianza totale all'interno di ogni cluster.
Oltre a minimizzare questa varianza, si vuole anche andare a massimizzare la varianza intercluster, cioè quella tra elementi appartenenti a cluster differenti.
L'output di questa fase è un dendogramma simile a quello di figura \ref{fig:dendogramma}
\begin{figure}[H]
    \centering
    \includegraphics[scale=0.5]{img/chap_2/PCA_CLUSTERING_5.PNG}
    \caption{output Clustering}
    \label{fig:dendogramma}
\end{figure}
\subsection{PCA \& clustering con 4 componenti}
Andando ad effettuare la clusterizzazione con 4 componenti principali, la quantità di varianza che varia in base al numero di cluster scelti è la seguente:
\begin{itemize}
  \item \textbf{5 Cluster:} Varianza pari a 63.96 \% con una perdita del 17,044\% rispetto alla varianza post PCA di 81.004\%
  \item \textbf{30 Cluster:} Varianza pari a 80.42 \% con una perdita del 0.58\% rispetto alla varianza post PCA di 81.004\%
  \item \textbf{50 Cluster:}Varianza pari a 80.63 \% con una perdita del 0,374\% rispetto alla varianza post PCA di 81.004\%
\end{itemize}
Qui di seguito è presentato il dendrogramma, utile per comprendere quanta varianza si perda o si mantenga al variare del numero di cluster.

%inserire dendogramma

\subsection{PCA \& clustering con 5 componenti}
Andando ad effettuare la clusterizzazione con 5 componenti principali, la quantità di varianza che varia in base al numero di cluster scelti è la seguente:
\begin{itemize}
  \item \textbf{5 Cluster:} Varianza pari a 59.02 \% con una perdita del 26,753\% rispetto alla varianza post PCA di 85.773\%
  \item \textbf{30 Cluster:} Varianza pari a 84.92 \% con una perdita del 0,853\% rispetto alla varianza post PCA di 85.773\%
  \item \textbf{50 Cluster:}Varianza pari a 85.32 \% con una perdita del 0,453\% rispetto alla varianza post PCA di 85.773\%
\end{itemize}
Qui di seguito è presentato il dendrogramma, utile per comprendere quanta varianza si perda o si mantenga al variare del numero di cluster.

\subsection{PCA \& clustering con 6 componenti}
Andando ad effettuare la clusterizzazione con 6 componenti principali, la quantità di varianza che varia in base al numero di cluster scelti è la seguente:
\begin{itemize}
  \item \textbf{5 Cluster:} Varianza pari a 56.66 \% con una perdita del 33,21\% rispetto alla varianza post PCA di 89.870\%
  \item \textbf{30 Cluster:} Varianza pari a 87.45 \% con una perdita del 2,42\% rispetto alla varianza post PCA di 89.870\%
  
  \item \textbf{50 Cluster:}Varianza pari a 88.38 \% con una perdita del 1,49\% rispetto alla varianza post PCA di 89.870\%

\end{itemize}
Qui di seguito è presentato il dendrogramma, utile per comprendere quanta varianza si perda o si mantenga al variare del numero di cluster.

\section{Conclusioni}
Analizzando i valori del precedente paragrafo sintetizzati nei seguenti grafici \ref{fig:varianzaSpiegata} e \ref{fig:varianzaPersa}.
Da questi possiamo decidere di selezionare 5 componenti principali e 30 cluster spiegando quindi l'84.92\% di varianza
\begin{figure}[H]
    \centering
    \includegraphics[scale=0.5]{img/chap_2/Varianza.jpeg}
    \caption{Varianza spiegata}
    \label{fig:varianzaSpiegata}
\end{figure}

\begin{figure}[H]
    \centering
    \includegraphics[scale=0.5]{img/chap_2/Loss.jpeg}
    \caption{Varianza persa}
    \label{fig:varianzaPersa}
\end{figure}
\section{Workload sintetico}
Tramite l'analisi effettuata è stato possibile realizzare il workload sintetico seguendo le conclusioni di cui sopra. Da 24 colonne e 3000 righe possiamo disporre di un workload di 24 colonne e solamente 30 righe.
Tale workload è mostrato nella figura \ref{fig:workload}
\begin{figure}[H]
    \centering
    \includegraphics[scale=0.4]{img/chap_2/Workload.png}
    \caption{Workload sintetico}
    \label{fig:workload}
\end{figure}